The network attacks attempted take advantage of security vulnerabilities in the network protocols that are used for computer system communication. The main problem for network communication is that it is unknown whether a device is malicious or not. Even genuine network users can be compromised and used for an attack against others. The attacks discussed are usually used by malicious actors when expanding upon a discovered vulnerability. This is because the most potent of attacks like ARP spoofing and session hijacking require a compromised local network host or a node that can act as a man-in-the-middle between two targets. Attackers can utilise two methods for exploiting network vulnerabilities: passive and active attacks. 
\\
\\
Passive attacks rely on snooping on unencrypted network traffic. They are hard to detect but can also be mitigated with many methods. Using encryption at some level of the network stack to secure critical information would be a standard method. IPSec and TLS are two protocols designed to offer modern solutions to encrypted network communication. Some of the attempted network attacks relied on having the attacker interface in promiscious mode in order to snoop other hosts' packets. Using a switched network with a link layer routing device would have prevented the snooping as packets would not be routed to the attacker. This however is less effective especially if a real man in the middle is compromised like the network gateway. 
\\
\\
Unlike passive attacks active attacks would have hosts send maliciously crafted packets over the network to exploit protocol security weaknesses. Spoofing would refer putting erroneous information inside packets. This provides an attacker with grater capabilities but would generate suspicious network activity that can be monitored, identified and could lead to identifying the security vulnerability in the software that can be fixed. This means that if a node like the network gateway is compromised it may not be beneficial to launch denial of service attacks like the TCP reset but instead try use more powerful exploits like the TCP session hijacking. To prevent this in the event of a compromised man in the middle the minimum an application should do is provide message authenticity and integrity checks. The telnet command injection demonstrates how powerful an attack can be against protocols that do not do at least message authenticity provisions. To address denial of service active attacks network routers should discard identifiably spoofed packets like impossible IP addresses for the sub-network. Attacks like the ICMP redirect that exploit powerful insecure features are also mitigated by software configuration options.
\\
\\
The nature of the network attacks discussed lies in weaknesses and exploit of trust in the used network protocols. As such these attacks would remain relevant while the protocols are in use. Since these protocols underpin the structure of the current Internet and networking it is unavoidable to use vulnerable protocols. The good side of this is that all possible weaknesses are known and can be mitigated by the software implementing the network protocols in use and network configurations. OWASP \cite{owasp_network} lists insecure network services are one of the most prevalent flaws in Internet of Things devices. It also suggests that abnomal network traffic that would be generated during a spoofing attack be detected and filtered as well as ensuring applications can not be used for a denial of service attack against other users.
\\
\\
As a final thought it is interesting to note the application of services that do not fully comply with the established protocols in order to prevent attacks. This is the case for evading Chinese internet censorship. The paper \cite{china} discusses non-TCP compliant servers that can be resilient to spoofed TCP reset and session hijacking packets.
