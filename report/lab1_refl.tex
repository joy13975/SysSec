% A reflective part, in which you critically reflect on the circumstances (both technical and organisational) that can give rise to the relevant vulnerabilities and potential mitigations. This can include e.g. mention of real instances of these attacks that you have researched, comparison to other attacks, discussion of relevant standards and recommendations (CERT, OWASP etc.). You can also discuss topics relating to the vulnerablities that you exploited, including: their potential risk and impact in real systems; why they (still) exist; which straightforward attempts at mitigation do not work and why not; how you would sucessfully reduce or eliminate these vulnerabilities.

%Cite occurences that brought people's attention to the exploit
Format string bugs were first reported in 1989 \cite{Miller1990} and gained more public attention a decade after when a security audit of the ProFTPD daemon unveiled the vulnerability \cite{tymm1999}. Today, it has been widely documented \cite{fsa_owasp, Weitz2014, arbaugh1997automated, scut2001}, with numerous protection techniques proposed, such as \cite{Shankar2001, cowan2002}. The Common Weakness Enumeration (CWE) dictionary keeps six records of real incidents where format strings were exploited \cite{fsa_cwe}, while Kilic et al counted around a dozen per year up to 2013 \cite{Kilic2015}. The CWE rates this exploit as very high likelihood, with common consequences of confidentiality loss and arbitrary code execution. These are serious impacts to system security considering that so many programs are written C and C++, which are quoted as \emph{often} suffering from format string attacks \cite{fsa_cwe}.

In this Lab, we have investigated the technical aspect that enables format string attacks to still be viable, albeit with some help to discover our address of interest. Although we have only shown the writing and reading of memory addresses, more sophisticated attacks are possible through meticulously crafted attacks. For example, arbitrary code execution and privilege escalation can be achieved by writing shell code onto the stack. One such incident is the CVE-2002-0573, whereby a call to the syslog function with user input allowed the execution of shell code in the RPC wall daemon for Solaris 2.5.1 through 8 \cite{cve-solaris2002}. Other cases include format strings taking down Splinter Cell gaming servers \cite{Auriemma} and Windows FTP server sin 2004 \cite{Winter-Smith2004}.

Fortunately, the No-eXecution (NX) memory protection was able to greatly thwart such efforts, and was popularized circa 2004. Today, most operating systems have ASLR and NX turned on by default, which makes address disclosure more difficult. We observed in Task 3 that in order to cope with changing addresses, more advanced techniques were required to insert the address bytes. Furthermore, performing the exploit manually became impractical as the addresses change in each execution. According to CWE, format string vulnerabilities have become rare today because of the ease of detection and the fact that the misuse of the \emph{printf()} family is now uncommon \cite{fsa_cwe}.

In the bigger picture, the format string attack is somewhat similar to SQL injection, whereby unsanitized input gets interpreted as part of the command. The SQL injection has been eliminated altogether by using parameterised statements. An approach to achieve something similar in general-purpose programming languages would be to restrict format strings to compile-time constant values, and only allow variable strings after the first argument. Moreover, since the \emph{printf()} family of functions are well known and rather static in nature, compilers could enforce a failure when substitution tokens and substitution variables mismatch (i.e. perform compile time lexical analysis \cite{alanpscan}). Another method is to implement type-checking as suggested by \cite{Weitz2014}. Although these methods slightly reduce programming freedom (e.g. no overloading of \emph{printf()} with single argument), the gain in security is well worth it.

On the other hand, good coding practice should be propagated from an early stage in educating programmers. It appears that most formal education skip the details on such "mundane" programming utilities. It would go a long way with a simple demonstration that \emph{printf(input)} is exploitable, while \emph{printf("\%s", input)} is not.

The Computer Emergency Response Team (CERT) for the Software Engineering Institute (SEI) recommands to "never call a formatted I/O function with a format string containing a tainted value" \cite{burch_cert}, with very detailed examples of standard and poor practices. Tainted value here refers to any data source that is not sanitized. It should be noted that the non-trivial process of string sanitization and its lack of reliability mean that passing sanitized user input directly as the format string is still not recommendable.