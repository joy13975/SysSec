% A reflective part, in which you critically reflect on the circumstances (both technical and organisational) that can give rise to the relevant vulnerabilities and potential mitigations. This can include e.g. mention of real instances of these attacks that you have researched, comparison to other attacks, discussion of relevant standards and recommendations (CERT, OWASP etc.). You can also discuss topics relating to the vulnerablities that you exploited, including: their potential risk and impact in real systems; why they (still) exist; which straightforward attempts at mitigation do not work and why not; how you would sucessfully reduce or eliminate these vulnerabilities.

%Is this exploit still relevant today and why/why not?
%Cite occurences that brought people's attention to the exploit

%Are there new problems relating to this exploit today?
%Cite potential risks to real systems and what can be done about it

%Introduce related vulnerabilities or variants for research points

%Finally, close with best practices to mitigate the vulnerability
%Explain a simple method that doesn't work perfectly
%Explain a standard way to eliminate/minimise the risk of the vulnerability

Buffer Overflow vulnerabilities are some of the most dangerous security risks. The Open Web Application Security Project \cite{owasp} treats them as very high severity at high likelihood of exploit. At minimum such vulnerabilities can result is software crashes and denial of service attacks. Their very high severity comes when a buffer overflow vulnerability results in arbitrary code execution. This can subvert any security measure and can compromise the entire computer that is executing the vulnerable software. Thankfully this vulnerability is eliminated by higher level languages that do not allow direct memory access and enforce array boundary checking. This means that much of software running on personal computers can easily eliminate the risk of this vulnerability. However system or high-performance software that rely on the features of programming in C/C++ or Assembly would still be exposed to the risk. \\ \\
The Blaster Worm\cite{balster_cert} is a good example of the severity of buffer overflows. It garnered media attention in 2003 as it was able to quickly spread throughout the internet without user interaction. The worm used a buffer overflow in Microsoft's RPC protocol to compromise machines. It demonstrates the power of remote code execution that results from exploiting buffer overflows. From taking a historic look into attacks using this threat it can be noted that buffer overflows were significantly more prevalent before the introduction of Windows Vista that includes bot ASLR and non-executable stack as standard mitigation security features. Nevertheless buffer overflows can still be taken advantage of especially in the sphere of embedded and Internet of Things devices. Those computer systems are usually limited in function and require the use of the C/C++ programming languages. In addition consumer focused devices target short time to market deadlines that put pressure on the software developers and heighten the probability of introducing vulnerabilities. \\ \\
From the investigation of the available defences against buffer overflows it can be noted that the possibility of exploitation can be significantly reduced by employing all available defences, but it does not completely eliminate the risk. If a system uses a position independent executable with ASLR in 64-bit addressing with non-executable stack and stack guards successful exploitation would likely be dependent on multiple vulnerabilities in the software. The best available solution to buffer overflows would be to use a managed programming language that eliminates the vulnerability. If a low-level language is required however this threat is likely impossible to eliminate. Defences against the vulnerability only mitigate risk of exploitation to reduce the likelihood of the vulnerability software development procedures must be implemented by the creator of the software. While automatic checking for known "bad" functions like \emph{strcpy} can be used to prevent the most basic buffer overflows, this vulnerability type is often hard to detect. Code review and security training should be required when working with programming languages that have potential to introduce buffer overflows. Even so the possibility of introducing a vulnerability or having a library dependency with a buffer overflow can not be completely eliminated. The severity of exploiting this vulnerability can be significant enough for software designers to consider automated patching functionality even with low likelihood of having a buffer overflow.

%end