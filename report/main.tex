\documentclass[12pt, onecolumn]{IEEEtran}

\usepackage[margin=25mm]{geometry}
\newcommand\tab[1][1cm]{\hspace*{#1}}
\setlength{\parindent}{0pt}

%break url line in references
\usepackage[hyphens]{url}
\usepackage{hyperref}
\hypersetup{breaklinks=true}
\urlstyle{same}

\usepackage{cite}

\usepackage{array}

\hyphenation{op-tical net-works semi-conduc-tor intro-duction high-light}

\begin{document}

COMSM1500 (Systems Security)\\
Coursework 1\\
\\
Students: cy13308 and rz13310 \\
\\
Contribution: \\
\tab[1cm] cy13308: set up own VM; completed task 1, attempted task 2, report on task 1\\
\tab[1cm] rz13310: set up own VM; attempted task 1, fixed and finished task 2, report on task 2
\\
\\
Overal contribution: cy13308 50\%, rz13310 50\%
\newpage

\section{Overview}
%One page intro/overview here explaining what this report contains, briefly describe procedures (SEED VM etc) and our conclusion
This coursework demonstrates network vulnerabilities in each of the four Internet protocol suite layers: from  ARP spoofing (Link Layer) and ICMP redirect (Internet Layer) to TCP reset and hijack (Transport Layer) and SQL injection (Applcation Layer). Multiple SEED labs virtual machines were used to form a small offline network in our experiments. This report assumes that the reader knows how to use basic Linux commands and some php coding. All source codes referred to in this report are available in the viritual machine image itself.

In each technical section, the vulnerabilities are first introduced by prodiving the respective context. We then craft attacks to break each of the four networking mechanisms, and discuss modern approaches to defend against these attacks. Finally, we brief background review is given.
\newpage

\section{Technical Report}
\subsection{Task 1}
\newpage
\subsection{Task 2}
\newpage

\section{Reflective Report}
\subsection{Task 1}
\newpage
\subsection{Task 2}

% \appendices
% \section{Proof of the First Zonklar Equation}
% Appendix one text goes here.

% you can choose not to have a title for an appendix
% if you want by leaving the argument blank
% \section{}
% Appendix two text goes here.

% \bibliographystyle{IEEEtran}
% \bibliography{mendeley_library}
\end{document}

