\documentclass{article}

\usepackage[margin=25mm]{geometry}
\newcommand\tab[1][1cm]{\hspace*{#1}}
\setlength{\parindent}{0pt}

%break url line in references
\usepackage[hyphens]{url}
\usepackage{hyperref}
\hypersetup{breaklinks=true}
\urlstyle{same}

\usepackage{cite}

\usepackage{array}

\hyphenation{op-tical net-works semi-conduc-tor intro-duction high-light}

\begin{document}

COMSM1500 (Systems Security)\\
Coursework 1\\
\\
Students: cy13308 and rz13310 \\
\\
Contribution: \\
\tab[1cm] cy13308: set up own VM; completed task 1, tried task 2, report on task 1\\
\tab[1cm] rz13310: set up own VM; tried task 1, fixed and finished task 2, report on task 2
\\
\\
Overal contribution: cy13308 50\%, rz13310 50\%

\newpage

test cy13308
\newpage

test rz13310

% \appendices
% \section{Proof of the First Zonklar Equation}
% Appendix one text goes here.

% you can choose not to have a title for an appendix
% if you want by leaving the argument blank
% \section{}
% Appendix two text goes here.

% \bibliographystyle{IEEEtran}
% \bibliography{mendeley_library}
\end{document}

