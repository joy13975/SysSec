% A technical part, describing what you did and how/why it works. Imagine that you are writing for another student who has not taken this unit: they should be able, after reading your technical part, to both reproduce your attacks by following the steps you give and to have a basic understanding of what is happening.

%short intro on format strings

%what could go wrong

%what we did

%how/why did what we did work
\subsection{Task 1}
The first task of the second lab ask us to exploit the buffer overflow of function \emph{bof} in the provided vulnerable program. This is a classic technique of overrunning a local buffer in a function stack frame to overwrite the saved return address. When the function eventually tries to return the execution flow is diverted to the overwritten address which could allow an attacker to execute arbitrary code with the privileges  of the exploited process.

To create this exploit our first goal was finding the offset in our input that overwrites \emph{bof}'s return address. This means finding the beginning of the overflown buffer in relation to the function's stack frame. By disassembling the \emph{stack-root} program we find in instruction 6 that this is \emph{-0x20(\%ebp)} or 32 byte before \emph{\%ebp}. This instruction sets the buffer address as an argument to \emph{strcpy()}. After these 32 bytes the stack frame contains 4 bytes of the saved \emph{\%ebp} followed by 4 bytes of the return address.

\begin{minipage}{\linewidth}
\begin{lstlisting}[caption={bof function disassembly},
frame=single]
08048484 <bof>:
8048484:	55                   	push   %ebp
8048485:	89 e5                	mov    %esp,%ebp
8048487:	83 ec 38             	sub    $0x38,%esp
804848a:	8b 45 08             	mov    0x8(%ebp),%eax
804848d:	89 44 24 04          	mov    %eax,0x4(%esp)
8048491:	8d 45 e0             	lea    -0x20(%ebp),%eax
8048494:	89 04 24             	mov    %eax,(%esp)
8048497:	e8 e4 fe ff ff       	call   8048380 <strcpy@plt>
804849c:	b8 01 00 00 00       	mov    $0x1,%eax
80484a1:	c9                   	leave
80484a2:	c3                   	ret
\end{lstlisting}
\end{minipage}

We make an exploit that consists of 36 junk bytes 0xAA followed by the target address in little-endian followed by the 24 bytes of provided shellcode. For successful execution the target address needs to point at the beginning of the shellcode. We find this address by executing the \emph{stack-root} program in \emph{gdb} and setting the \emph{ret} instruction of \emph{bof} as a breakpoint. We take note of \emph{\%esp} which in our build is \emph{0xBFFFF4BC}. At that instruction \emph{\%esp} would point to the return address on the stack and if the shellcode is located just after that return address, a pointer to it would be 4 bytes higher than the obtained value - \emph{0xBFFFF4C0}. Using that as our target return address we successfully execute the exploit in \emph{gdb}. Since running the program outside \emph{gdb} the stack has a slight offset this exploit fails. To make our exploit resistant to stack offset variations we prepend the shellcode with 256 bytes of \emph{nop} instructions and increment the return address by 128 setting it to \emph{0xBFFFF540}. This gives the exploit operability when the stack offset is within 128 bytes of the stack offset under \emph{gdb}.

\subsection{Defence mechanisms}
\subsubsection{ASLR}
The first defence technique to consider against this exploit is Address Space Layout Randomization or ASLR. This is an Operating System security feature implemented by the program loader that randomises the allocations for the heap, stack and loaded libraries. It does not prevent the actual buffer overflow but can mitigate it's exploitation potential. Since the crafted exploit needs to know the approximate location of the shellcode and that is located on the now randomised stack. With ASLR we can no longer obtain the stack location through seeing the memory maps of a single program instance because different executions allocate the stack at random location. By trying enough times it is possible to get successful exploitation if the random stack coincides with the observed stack location on creating the exploit. This is more likely to be feasible on 32-bit environments as they only offer 16 bits of randomisation which is significantly smaller than the available randomisation on 64-bit platforms. However there exist several methods to bypass this protection. A popular one is using the executable code and global variable space  which is the remaining known data fixed in the executable's address space. This is a very powerful technique but can be preventing by adopting compilation of position independent code (PIC). With PIC even the executable code's addresses can be randomised. This leaves the attacker only with finding an address leak as a viable attack on ASLR. If however such an information leak exists then it renders ASLR completely ineffective as the attacker can compute the exact offset of the leaked segment.\\
\subsubsection {Stack Guard}
When enabled the stack guard \emph{gcc} option generated the result in \lstref{lst:sg_out} regardless of wheather the \emph{-g} option was present or not. 
\begin{lstlisting}[caption={Stack Gard enabled buffer overflow output},
label={lst:sg_out},
language=,
frame=single]
*** stack smashing detected ***: ./stack-stprot terminated
Segmentation fault (core dumped)
\end{lstlisting}
By looking at function disassembly with Stack Guard enabled we observe that the method consists of additional prologue and epilogue code in function generation. The prologue copies a random value called canary from protected memory and stores it between the function's local variable stack space and the saved base pointer. If a buffer overflow then occurs in this function then it must overwrite the canary before the saved return address. The epilogue tests if the canary is different than the initial value before reaching a \emph{ret} instruction. If a change has occurred  a dynamically linked handler is called. This handler evidently generates the output in \lstref{lst_sg_out} and terminates execution. This defence also does not prevent the overflow it only detects and attempts to mitigate the severity of the threat.\\
While troublesome for straightforward exploitation the weakness of this method is in the fact that the canary does not protect the local stack space variables and is only checked on function exit. Thus code between the occurrence of the overflow and the function return can be abused. As an example the function in \lstref{lst:sg_example} can be exploited by using the \emph{tmp\_key} overflow to modify the \emph{data} pointer. The second string copy could then write data to arbitrary memory locations. An exploit can either start from the return address without modifying the canary or it can modify the pointer to the dynamically linked stack thrashing handler. G. Richarte\cite{Stackgua58} explores similar techniques for circumventing Stack Guard type protections.

\begin{lstlisting}[caption={Stack Gard vulnerable function},
label={lst:sg_example},
language=C,
frame=single]
void vuln(char* name,char* val)
{
    char *data = malloc(strlen(val)+1);
    char tmp_key[24];
    
    strcpy(tmp_key,name);
    
    strcpy(data,val);
    
    //do work e.g. build dictionary entry

    return;
}
\end{lstlisting}
\subsubsection { Non-Executable Stack }
When compiled with default option of non-executable stack region the memory pages allocated for stack space are set without execute permissions which means that an exception is generated if the CPU tries to read code from those memory locations. Running the exploit with this feature causes it to generate a Segmentation Fault. This is yet another mitigation technique. It is also easy to circumvent since stack data is still under attacker control and the exploit would just need to use portions of the executable's code segment. The attack can be identical to the one discussed against non-PIE executables with ASLR. \\
\subsection {Extra Credit}
\subsubsection{ Real and Effective UIDs }
To set the real UID to to 0(root) a \emph{setuid} system call needs to be performed before the shellcode that executes the terminal. The code in \lstref{lst:suid_asm} executes the system call by setting \emph{\%eax} to \emph{0x17} and \emph{\%ebx} to \emph{0x0}. This code in hex converts to the 8 bytes in \lstref{lst:suid_hex} that are prepended to the 24 byte shellcode in the exploit. Runnig this modified exploits results in successful modification of the UID to root before the execution of the shell.
\begin{lstlisting}[caption={Setuid assembly syscall},
label={lst:suid_asm},
language=,
frame=single]
xor %ebx, %ebx
xor %eax, %eax
mov %al , 0x17
int 0x80
\end{lstlisting}
\begin{lstlisting}[caption={Setuid disassembled},
label={lst:suid_hex},
language=,
frame=single]
31 DB 31 C0 B0 17 CD 80
\end{lstlisting}
\subsubsection { Defeating ASLR and non-executable stack }
An exploit based on ROP gadgets\cite{rops} is developed for exploiting a slightly modified version of \emph{stack.c}. To make it The modification includes a \emph{printf} call with the string \emph{"sh"}. The source is provided in the \emph{stack-help.c} file which is compiled with only stack guard disabled. This is to simplify the exploit but is unnecessary if a further technique of custom stack creation is used. The exploit uses the fact that the code segment of the executable is always in know address spaces and sets up a stack that redirects execution to sequences of instructions that end with an \emph{ret} instruction. Thanks to the variable instruction size CISC architecture of x86 it is also possible to jump in the middle of instructions and reinterpret the data as different instructions. A tool called \emph{ropper}\cite{ropper} is used to list useful instruction sequences called ROP gadgets. When a function to a dynamically linked library is called the code calls the fixed PLT entry of the function. This PLT entry is responsible for loading the location of the dynamic function into the Global Offset Table (GOT) and then branching to the stored entry in that table. The goal of the exploit would be to overwrite the \emph{printf} entry with the \emph{system} function offset and then set up a \emph{printf} call with the string \emph{"sh"} as argument effectively calling \emph{system("sh")}. From disassembling the executable we obtain the printf@PLT address that is \emph{0x08048390} and the GOT entry \emph{0x0804a000}. We also use \emph{nm} to view the difference of offset between \emph{printf} and \emph{system} in \emph{libc}.
\begin{lstlisting}[caption={Libc symbols},
label={lst:libc_sym},
language=,
frame=single]
nm -D /lib/i386-linux-gnu/libc.so.6 
...
0003f430 W system
...
0004ced0 T printf
...
\end{lstlisting}
The difference between \emph{printf} and \emph{system} is \emph{0xffff2560}, which is important to be negative as it will be included in the exploit which needs to include no null bytes. The ROP gadgets found in the code that would be used ar listed in \lstref{lst:gadgets}. 
\begin{lstlisting}[caption={List of used gadgets},
label={lst:gadgets},
language=,
frame=single]
Gadget A:  
	0x0804847e: 
		add dword ptr [ebx + 0x5d5b04c4], eax; ret;

Gadget B:  
	0x08048593: 
		mov eax, dword ptr [esp + 0x34]; 
		mov dword ptr [esp + 4], eax; 
		call dword ptr [ebx + esi*4 - 0xe0]; 
		add esi, 1; cmp esi, edi; 
		jne 0x588; add esp, 0x1c; 
		pop ebx; pop esi; pop edi; pop ebp; ret; 

Gadget C:  
	0x080485ac: 
		pop ebx; pop esi; pop edi; pop ebp; ret; 

Gadget D:
	0x08048378: 
		pop ebx; ret; 
\end{lstlisting}
Gadget C and D allow direct control over \emph{\%ebx} \emph{\%esi} \emph{\%edi} and \emph{\%epb}, Gadget A allows for modification of an arbitrary memory address. We would use that gadget to add the difference between \emph{printf} and \emph{system} to the \emph{printf} GOT entry. To control \emph{\%eax} we need to use gadget B which has the caveat of having to call a function that does not modify \emph{\%eax} and needs \emph{\%esi+1 == \%edi}. To get a function address that does not modify \emph{\%eax} we use the dynamic linking entry for function \emph{\_fini}. Which executing the program in \emph{gdb} and listing the DYNAMIC section lets us determine that is stored in \emph{0x08049f3c}. Furthermore \emph{\_fini} needs a non-zero value in location 0x0804a024 in order to do nothing. We can use gadget A to add the initial non-zero \emph{\%eax} to that location. Finally we determine to set \emph{\%esi=0x01020101} \emph{\%edi=0x01020102} and \emph{\%ebx=0x03fc9c18} in order to satisfy all condition for executing this complex gadget. Finally we determine the location of the string \emph{"sh"} to be \emph{0x0804862a} and we note that we should execute a \emph{printf} call to initialise the GOT entry. \\
the final Exploit is executed in this order :
\begin{enumerate}
\item call printf@PLT with "sh" argument
\item use gadget D to pop the leftover "sh" pointer
\item use gadget D to set \%ebx to 0xaaa99b60 which used by gadget A would modify 0x0804a024
\item use gadget A to modify 0x0804a024 to non-zero
\item use gadget C to set \%esi = 0x01020101 \%edi = 0x01020102 \%ebx = 0x03fc9c18
\item use gadget B to set \%eax = 0xffff2560 and \%ebx = 0xaaa99b3c which used in gadget A would modify 0x0804a000
\item use gadget A to modify the printf GOT entry 0x0804a000 with its diference to system
\item call printf@PLT with "sh" as argument which effectively calls system("sh")
\end{enumerate}
Taking into account the specifics of calling the gadgets and PLT functions the exploit should look like \lstref{lst:aslr_hex}. All the 32-bit data should be encoded in Little-endian. While this exploit relies on the \emph{"sh"} string being present a modified version could create it in global variable memory by a series of \emph{strcpy} instruction from program memory for each byte. In fact with this technique a custom stack can be created and then pivoted to using a \emph{leave} \emph{ret} instruction sequence.
\begin{lstlisting}[caption={ASLR exploit data},
label={lst:aslr_hex},
language=,
frame=single]
| 36 junk bytes to reach saved return
| printf@PLT address
| gadget D address
| "sh" address
| gadget D address
| 0xaaa99b60 ( new ebx )
| gadget A address
| gadget C address
| 0x03fc9c18 (new ebx )
| 0x01020101 (new esi)
| 0x01020102 (new edi)
| 4 junk bytes (new ebp)
| gadget B address
| 28 junk bytes 
| 0xaaa99b3c (new ebx)
| 4 junk bytes (new esi)
| 4 junk bytes (new edi)
| 4 junk bytes (new ebp)
| gadget A address
| printf@PLT address
| 0xffff2560 (new eax) for gadget B
| address of string "sh"
\end{lstlisting}

